\chapter*{Abstract}
\subsubsection*{\hsmatitelen}
Today, more and more formerly analogue physical entities are now being digitized and connected to the internet, adding to the \enquote{Internet of Things}. However, the wide variety in appliances poses a potentially wide attack surface for malicious actors. To address this risk that these so-called smart devices pose to parties that employ them, security researchers and penetration testers examine and test their security implementation. The need arises for a modular proxy application that allows to test the heterogeneous landscape of communication protocols used in IoT applications. The goal of this thesis is to conceptualize a design for such an application, realize a prototypic implementation thereof and evaluate its usefulness. Quantitative results are a documentation of the problem space, an abstract design concept and sets of development challenges and lessons learned. 

\subsubsection*{\hsmatitelde}
Als Konsequenz der voranschreitenden Digitalisierung werden ehemals analoge Geräte zunehmend digitalisiert und somit Teil des \enquote{Internets der Dinge} (\enquote{Internet of Things}, IoT). Dabei stellt jedoch die große Bandbreite an Anwendungen eine potenziell große Angriffsfläche für Angreifer dar. Um diesem Risiko, das die sogenannten \enquote{smarten} Anwendungen gegenüber ihren Betreibern darstellen, zu begegnen, untersuchen und überprüfen Sicherheitsforscher und Penetrationtester deren Sicherheitsarchitekturen. Daraus erwächst ein Bedarf an einer modularen Proxy-Anwendung, die sie dabei unterstützt, die heterogene Verwendung von Kommunikationsprotokollen in IoT-Anwendungen zu beherrschen. Ziel dieser Arbeit ist die Konzeption eines Softwareentwurfs für eine solche Anwendung und deren prototypische Umsetzung sowie eine Bewertung ihrer Nützlichkeit. Quantitative Ergebnisse sind die Dokumentation des Problemstellung, ein abstraktes Entwurfskonzept und eine Reihe von Herausforderungen bei der Entwicklung sowie daraus gewonnene Erkenntnisse.