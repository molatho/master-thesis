\chapter{Theoretical Background}
\label{chap:theoretical-background}
This chapter provides an overview of the technologies and concepts referred to in subsequent chapters.
Starting with section \ref{sec:computer-networks}, essential concepts of computer communication in networks will be presented and examined, covering the concept of network layers, intercepting of communication between two parties and analysis of transferred data.
Building upon these fundamentals, section \ref{sec:internet-of-things} introduces the fields of use of \ac{IoT} applications, common architectures used today to implement them and popular protocols they make use of. Lastly, it will discuss security considerations important to \ac{IoT} applications.
After that, section \ref{sec:information-security} will provide insights into relevant concepts and the practices used and applied in information security. It covers key concepts and legal considerations, integration of information security in software development and common practices and methods involved. 

\section{Computer Networks}
\label{sec:computer-networks}
\subsection{Network Layers}
\emph{TBD} %TODO
\ac{TCP}

\subsection{Proxying Network Traffic}
\emph{TBD; planned:} %TODO
\begin{enumerate}
    \item Definition; Working Principle
    \item Use Cases
    \item Abuse Cases
\end{enumerate}

\subsection{Deep Packet Inspection}
\emph{TBD} %TODO

\section{(Industrial) Internet of Things}
\label{sec:internet-of-things}
\subsection{Fields of Use}
\subsection{Application Architectures}
\subsection{Common Protocols}
\label{sec:iot-common-protocols}
Building up on pre-existing network infrastructure and in order to meet requirements specific to individual fields of use and use-case scenarios, the landscape of \ac{IoT} attends with a great variety of \emph{communication protocols} (further used to refer to both transport and application protocols). This section will provide a brief overview of the working principles, use cases and history of some protocols commonly used in \ac{IoT} and \ac{IIoT} applications today.
\paragraph{\ac{HTTP}} \emph{TBD} %TODO
\paragraph{\ac{WS}} \emph{TBD} %TODO
\paragraph{\ac{MQTT}} \emph{TBD} \ac{AWS} \ac{IoT}%TODO
\paragraph{Modbus \ac{TCP}} \emph{TBD} %TODO
\paragraph{Profibus/Profinet} \emph{TBD} %TODO
\paragraph{\ac{OPCUA}} \emph{TBD} %TODO

\subsection{Security Considerations}

\section{Information Security}
\label{sec:information-security}
\emph{TBD} %TODO
\subsection{Key Concepts} % Schutzziele
\subsection{Legal Background} % Datenschutz, Whitehats/Blackhats, 0days?
\subsubsection{Compliance}
\subsubsection{Data Protection}
\subsection{Integration in Software Development}
\subsubsection{Traditional Approaches}
\subsubsection{Modern Approaches}
\subsection{Methodology}
\subsubsection{Risk Management}
\subsubsection{Incident Response}
\subsubsection{Reverse Engineering} %Funktionsweise/Abläufe Nachvollziehen
\subsubsection{(Physical) Penetration Testing} %Schwachstellen finden
\subsubsection{Source Code Audits}
\subsubsection{Application Configuration}