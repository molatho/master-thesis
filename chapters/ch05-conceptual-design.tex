% 4. Konzept (Kontext, Ablauf, Anforderungen [Interviews], Konzept [Architektur])
\chapter{Conceptual Design}
\label{chap:conceptual-design}
This chapter will detail the process of conceptualizing the design of the modular proxy application based on the results of the preceding chapter. First, the requirements are analyzed for their potential design implications in section \ref{sec:req-design-implications}. Afterwards the user interactions and domain entities identified in chapter \ref{chap:understanding-the-problem-space} are examined and broken down into communication flows between actors and systems in section \ref{sec:user-interactions-designing-workflow} and individual software components that complete the design are discussed in section \ref{sec:inferring-software-components}. Lastly, an overview of the complete design concept is given in section \ref{sec:abstract-design-concept}, discussing potential advantages and constraints.

\emph{Note: sections \ref{sec:req-design-implications} and \ref{sec:user-interactions-designing-workflow} should probably be merged as they overlap a lot} %TODO

\section{Requirements: Design Implications}
\label{sec:req-design-implications}
\emph{TBD} %TODO

\section{User Interactions: Designing the Intended Workflow}
\label{sec:user-interactions-designing-workflow}
\emph{TBD} %TODO

\section{Inferring Software Components}
\label{sec:inferring-software-components}
\emph{TBD} %TODO

\section{Summary: An Abstract Design Concept}
\label{sec:abstract-design-concept}
\emph{TBD} %TODO
