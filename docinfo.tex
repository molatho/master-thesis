% Titel der Arbeit auf Deutsch
\newcommand{\hsmatitelde}{Konzeption, Implementierung und Evaluation eines Machbarkeitsnachweises eines modularen Proxys zum Testen von Anwendungen im Internet der Dinge}

% Titel der Arbeit auf Englisch
\newcommand{\hsmatitelen}{Conception, Implementation, and Evaluation of a Proof of Concept of a Modular Proxy Application for Testing Internet of Things Applications}

% Weitere Informationen zur Arbeit
\newcommand{\hsmaort}{Mannheim}    % Ort
\newcommand{\hsmaautorvname}{Moritz Laurin} % Vorname(n)
\newcommand{\hsmaautornname}{Thomas} % Nachname(n)
\newcommand{\hsmadatum}{31.05.2021} % Datum der Abgabe
\newcommand{\hsmajahr}{2020} % Jahr der Abgabe
\newcommand{\hsmafirma}{NVISO GmbH, Frankfurt} % Firma bei der die Arbeit durchgeführt wurde
\newcommand{\hsmabetreuer}{Prof. Dr. Thomas Specht, Hochschule Mannheim} % Betreuer an der Hochschule
\newcommand{\hsmazweitkorrektor}{Pierre-Alain Mouy, M.Sc., NVISO GmbH} % Betreuer im Unternehmen oder Zweitkorrektor
\newcommand{\hsmafakultaet}{I} % I für Informatik oder E, S, B, D, M, N, W, V
\newcommand{\hsmastudiengang}{IM} % IB IMB UIB CSB IM MTB (weitere siehe titleblatt.tex)

% Zustimmung zur Veröffentlichung
\setboolean{hsmapublizieren}{true}   % Einer Veröffentlichung wird zugestimmt
\setboolean{hsmasperrvermerk}{false} % Die Arbeit hat keinen Sperrvermerk

% -------------------------------------------------------
% Abstract
% Achtung: Wenn Sie im Abstrakt Anführungszeichen verwenden wollen, dann benutzen Sie
%          nicht "` und "', sondern \enquote{}. "` und "' werden nicht richtig
%          erkannt.

% Kurze (maximal halbseitige) Beschreibung, worum es in der Arbeit geht auf Deutsch
\newcommand{\hsmaabstractde}{TBD.} % TODO

% Kurze (maximal halbseitige) Beschreibung, worum es in der Arbeit geht auf Englisch
\newcommand{\hsmaabstracten}{TBD} %TODO
